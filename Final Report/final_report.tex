% ****** Start of file apssamp.tex ******
%
%   This file is part of the APS files in the REVTeX 4.2 distribution.
%   Version 4.2a of REVTeX, December 2014
%
%   Copyright (c) 2014 The American Physical Society.
%
%   See the REVTeX 4 README file for restrictions and more information.
%
% TeX'ing this file requires that you have AMS-LaTeX 2.0 installed
% as well as the rest of the prerequisites for REVTeX 4.2
%
% See the REVTeX 4 README file
% It also requires running BibTeX. The commands are as follows:
%
%  1)  latex apssamp.tex
%  2)  bibtex apssamp
%  3)  latex apssamp.tex
%  4)  latex apssamp.tex
%
\documentclass[%
 reprint,
%superscriptaddress,
%groupedaddress,
%unsortedaddress,
%runinaddress,
%frontmatterverbose, 
%preprint,
%preprintnumbers,
%nofootinbib,
%nobibnotes,
%bibnotes,
 amsmath,amssymb,
 aps,
%pra,
%prb,
%rmp,
%prstab,
%prstper,
%floatfix,
]{revtex4-2}

\usepackage{graphicx}% Include figure files
\usepackage{float}
\usepackage{dcolumn}% Align table columns on decimal point
\usepackage{bm}% bold math
\usepackage{hyperref}% add hypertext capabilities
%\usepackage[mathlines]{lineno}% Enable numbering of text and display math
%\linenumbers\relax % Commence numbering lines

%\usepackage[showframe,%Uncomment any one of the following lines to test 
%%scale=0.7, marginratio={1:1, 2:3}, ignoreall,% default settings
%%text={7in,10in},centering,
%%margin=1.5in,
%%total={6.5in,8.75in}, top=1.2in, left=0.9in, includefoot,
%%height=10in,a5paper,hmargin={3cm,0.8in},
%]{geometry}

%% my own packages
\usepackage{siunitx}

\begin{document}

\preprint{APS/123-QED}

\title{Manuscript Title:\\with Forced Linebreak}% Force line breaks with \\
\thanks{A footnote to the article title}%

\author{Matthias Maile}
 \email{matthias.maile@kaist.ac.kr}
\affiliation{
  Korea Advanced Institute of Science and Technology
}

\date{\today}% It is always \today, today,
             %  but any date may be explicitly specified

\begin{abstract}
  This is the abstract
  An article usually includes an abstract, a concise summary of the work
  covered at length in the main body of the article. 
  \begin{description}
    \item[Usage]
      Secondary publications and information retrieval purposes.
    \item[Structure]
      You may use the \texttt{description} environment to structure your abstract;
      use the optional argument of the \verb+\item+ command to give the category of each item. 
  \end{description}
\end{abstract}

%\keywords{Suggested keywords}%Use showkeys class option if keyword
%display desired
\maketitle

\tableofcontents

\section{Introduction}
\label{sec:introduction}

\section{Previous work}
\label{sec:previous_work}
In this report I will cover the previous work only briefly since it has been covered in my midterm report already.
Firstly I will explain the Intelligent Driver Model by Treiber et al. In his work a car following model has been derived
which mimics a driver who is well aware of his surrounding. Afterwards I will give an overview over the lane changing
decision algorithm devoleoped by Kersting et al.

\subsection{The Intelligent Driver Model (IDM)}
\label{sec:idm}
The IDM is a car following model, meaning that a the accelearation of a car is a function of the cars velocity,
distance to the car in front and the velocity of the car in front:
\begin{equation}
  \dot v_\alpha = a \left[1 - \left(\frac{v_\alpha}{v_0^{(\alpha)}}\right)^\delta -
    \left(\frac{s^*(v_\alpha, \Delta v)}{s_\alpha}\right)^2
  \right]
  \label{eqn:idm:sprime}
\end{equation}
It consists of a acceleration term $a [1 - (v_\alpha / v_0^{(\alpha)})^\delta]$ and a term modelling the breaking
behaviour $-a (s^* / s_\alpha)^2$ which gets larger the closer a car drives to the car ahead. The breaking behaviour
can be understood as the ratio between the ``desired minimum gap``
\begin{multline}
  s^*(v, \Delta v) = s_0^{(\alpha)} + s_1^{(\alpha)} \sqrt{\frac{v_\alpha}{v_0^{(\alpha)}}} + T^{(\alpha)}v_\alpha
  \\
  + \frac{v\Delta v}{2 \sqrt{a^{(\alpha)} b^{(\alpha)}}}
\end{multline}
and the actual gap.

\subsection{Minimizing overall braking induced by lane changes (MOBIL)}
\label{sec:mobil}
As explained in the midterm report, there are models for lane change strategies. For this project the model 
\textit{Minimizing overall braking induced by lane changes} has been chosen \cite{MOBIL}. The model has been developed by
a similar group as the IDM and they proved that they worked well together. 

One key feature of MOBIL is that it contained two modes of driving, symmetric and assymetric laws with the left
lane as the default lane. This is perfect for our use case as these are the `US` law and Vienna Convention laws
respectively. In the following I will give an overview about the equations for each case.

\subsubsection{Symmetric driving law}
The more simple symmetric driving law, is modeled by the \textit{incentive criterion}
\begin{equation}
  \underbrace{\tilde{a}_c - a_c}_\text{driver}
  + p (
  \underbrace{\tilde{a}_n - a_n}_\text{new follower}
  +
  \underbrace{\tilde{a}_0 - a_0}_\text{old follower}
  )
  > \Delta a_\text{th}.
  \label{eqn:US_incentive}
\end{equation}
While the lower index specifies which car is meant by an expression, the tilde marks a variable as if it a proposed
lane change is accepted, e.g. $\tilde{a}_n$ is the acceleration of the new follower after the car has changed the
lane.

This condition also introduces 2 model parameters, the \textit{politeness factor} $p$ and the switching threshold
$\Delta a_\text{th}$. The former regulates how strong a driver should consider the advantage of a lane change for
the next neighbors. On a qualitative note, $p=0$ indicates a driver ignoring his surrounding, while $p>0$ can be
interpreted as altruistic driving behaviour. Secondly, the lane switching threshold forbids lane changes with
minimal gains.

\subsubsection{Asymmetric driving law}
The assymetric driving law is more complicated and more complex decision trees. There are two basic rules that
apply here:
\begin{enumerate}
  \item Passing rule: A car must not pass on the right-hand lane. This rule only applies when traffic is not
    congested. Congested traffic is defined as cars moving below a \textit{ciritical velocity} 
    $v_\text{crit} = 60 \,
    \text{km/h}$.
  \item Lane usage rule: A car should only move to the left for overtaking and treat the right lane as the default
    lane.
\end{enumerate}

The passing rule is implemented by using a new variable on the right lane of a lane change:
\begin{equation}
  a_c^\text{eur} = 
  \begin{cases}
    \text{min}(a_c, \tilde{a}_c) & \text{if } v_c > \tilde v_\text{lead} > v_{crit}, \\
    a_c                         & \text{otherwise}
  \end{cases}
\end{equation}
with $\tilde a_c$ for the acceleration on the left lane and $\tilde v_\text{lead}$ corresponding to the velocity of
the car in front on the left lane.
Using $a_c^\text{eur}$ for the dynamics on right-hand lane implements the logic for the critical behaviour around
$v_\text{crit}$.

To implement the lane usage rule, an assymetry gets added to the incentive criterion. When a change to a right hand
lane is proposed, the inequility reads
\begin{equation}
  L \rightarrow R: \quad
  \tilde a_c^\text{eur} - a_c + p (\tilde a_o - a_o) > \Delta a_\text{th} - \Delta a_\text{bias},
\end{equation}
for changing to the left this inequality becomes
\begin{equation}
  R \rightarrow L: \quad
  \tilde a_c - a_c^\text{eur} + p (\tilde a_n - a_n) > \Delta a_\text{th} + \Delta a_\text{bias}.
\end{equation}
A key difference to \autoref{eqn:US_incentive} is that not all neighbors get considered anymore, as the left lane
(so the new follower for left-bound changes and old follower for changes to the right) gets a priority. Treating
the right lane as the default lane is enforced by the bias $\Delta a_\text{bias}$. With 
$\Delta a_\text{bias} > \Delta a_\text{th}$, a free moving car will change to the right.


\section{Methods}
\label{sec:methods}
Solving this problem will consist of two parts: 
\begin{itemize}
  \item A traffic simulation implementing the IDM and MOBIL,
  \item a programm analysing the data generated by the simulation.
\end{itemize}
Those two components will be nearly independent from each other.

\subsection{Traffic Simulation}
\label{sec:traffic_simulation}
The programm which simulates traffic and generates the data will be written in C++ for performance reasons. To get
meaningfull results, a few hundred cars will be simulated and as soon as you implement lane changes $\mathcal{O}(n^2)$
parts start to apear to find new following cars (with $n$ being the amount of cars). The multi lane road is implemented
as a class, which stores the cars in a vector from the standart template library. As a time integration method currently
Euler`s method is used, which might be changes to a 4th order Runge Kutta in the future. The data will be stores in
comma-seperated value (CSV) files. Altough csv files are slower than binary data files, they have the advantage to be
human readable for debugging purposes. Even with a slow output and $\mathcal{O}(n^2)$ parts in the algorithms,
a few hundred cars can be simulated in realtime on a laptop. 

The only part where analysis and simulation overlap is when it comes to macroscopic observables: Eventhough the data 
analysis programm explained later can calculate these observables from the generated microscopic data, parts of it will
be implemented inside the simulation. This not only makes the calculation more stable since no information gets lost or
cut of before the calculation, but also make the data sharing between the two programms smaller and less prone to error.

\subsection{Data Analysis}
\label{sec:data_analysis}
For the data analysis we will use python. The data can be easily imported with pandas and visualized in matplotlib.
Since the simulation can store every location of a car at any time, even video animations can be created, making the
simulating part more transparent. Since we are transforming the large data set of many microscopic observables into a
few macroscopic, only simple analysis tools like curve fits and similar numerical methods can be used. For this part
numpy and scipy are two highly capable libraries to be used.


\section{Results}
\label{sec:results}

\section{Conclusion and Outlook}
\label{sec:conclusion}


\appendix

\section{Compiling and running the program}
\label{sec:manual_execution}
This is only a minimal guidance, for more information a direct contact to the author is recomended,
either via e-mail or by opening an issue on the
\href{https://github.com/dormail/IntelligentDriver}{github repository of this project}. Overall,
this project has only been run on 
\href{https://www.debian.org/}{Debian 11} and \href{https://getfedora.org/}{Fedora 36}. As the
author has no access to a Windows machine, no testing has been done or is currently planned. The best bet to
still run this project is to utilize the Windows Subsystem for linux (WSL2).

\subsection{Building the program}
The project uses \texttt{cmake} as a build system. The following commands shall be run to compile
and install it (it assumes an terminal with the project`s main directory as a working directory):
\begin{enumerate}
  \item \texttt{mkdir build} to create the \texttt{build} directory
  \item \texttt{cmake -B build .} to create the build files in the directory
  \item \texttt{cd build} to change the current working directory to the newly created build
    directory
  \item \texttt{make} for compiling
  \item \texttt{sudo make install} this will ask for your user credentials since it installs the
    program. If you do not trust this program you can also leave this step out. Then you will have
    to change the execution command when running it in the terminal or in the python script
    accordingly. 
\end{enumerate}
Now the program can be run by typing \texttt{microscopicIDM} into the command line, if you did not
install it you need to run \texttt{./microscopicIDM}. You can also add the executable program to the
path of the shell to run it, there are few guides on it in the internet.

\subsection{Running the program and giving instructions}
This program can be executed in the terminal and accepts unix-style command line arguments denoted
with a dash (\texttt{-}).

The simplest execution is simply running \texttt{microscopicIDM}, as it will only run with default
values. \texttt{microscopicIDM -h} will show a help page with command line commands you can add to
specify parameters, e.g. \texttt{microscopicIDM --road-length 5000} to change the road length to
$\SI{5000}{m}$. The most important is the \texttt{--output-csv <filename>} command, as it specifies
the file name and location for the CSV file containing the data.

\section{Differentiation from the previous work}
\label{sec:diff_previous}
During my presentation a question has been raised how my project differs from the previous work. The
biggest differences between MOBIL and this project is
\begin{itemize}
  \item The amount of lanes has been increased to 3
  \item The road is circular
\end{itemize}
While the former is a trivial difference, the second is not. 

Changing the road from an infinite road stretch to a circular one  not only significantly changes the
way data structures and algorithms have to be designed and selected, it also creates a state of
equilibrium. The researchers for the MOBIL publication used and infinite road and had measurements
at fixed locations, after around 5 to 10 kilometers. There was no test if there is a state of
equilibrium so early, but by having a simulated time intervall of over an hour, there results have
been more stable. 

At the same time, it can safe a lot of runtime as less cars are needed and a visual test showed that
for road lengths over 1 km the marginal behaviour stemming from the border can not be observed. A
script creating a clip animation is also given in the project.



% The \nocite command causes all entries in a bibliography to be printed out
% whether or not they are actually referenced in the text. This is appropriate
% for the sample file to show the different styles of references, but authors
% most likely will not want to use it.
\nocite{*}

%\bibliography{apssamp}% Produces the bibliography via BibTeX.
\bibliography{final_report}% Produces the bibliography via BibTeX.

\end{document}
%
% ****** End of file apssamp.tex ******

\section{Motivation and Goal of this project}
\label{sec:goal}
The following part is a refined version of the intruction given in the midterm report.

Analyzing traffic laws on a global scale, regarding the rules about passing cars they can be sorted into two categories: One candidate
is the driving law as it is implemented in many European countries, which are summed up in the United
Nations Vienna convention about road traffic \cite{vienna-treaty}. According to the traffic law as defined in this
convention a driver is obligated to stay as far from the
center of the road when he is not overtaking another driver and shall move towards the center when he is overtaking.
Opposed to this there are countries, like some US states, where a driver is free to choose the lane for any maneuver.
A direct implication is that a car can be legally passed on both sites. In Germany, where traffic law obeys the Vienna
convention, both the person passing on the right and the person being passed can actually be prosecuted (see §2 STVO in 
German law \cite{STVO2}).

With those two models being widely implemented, one can raise the question which one performs better.
Since the term ``performance`` is ambiguous, this project limits itself to the overall car throughput and travel time. 
So the question this work shall answer is:
\newline
How does the obligation to drive on a specific side of the road impact car thoughput and speed?


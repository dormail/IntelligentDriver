\section{Compiling and running the program}
\label{sec:manual_execution}
This is only a minimal guidance, for more information a direct contact to the author is recomended,
either via e-mail or by opening an issue on the
\href{https://github.com/dormail/IntelligentDriver}{github repository of this project}. Overall,
this project has only been run on 
\href{https://www.debian.org/}{Debian 11} and \href{https://getfedora.org/}{Fedora 36}. As the
author has no access to a Windows machine, no testing has been done or is currently planned. The best bet to
still run this project is to utilize the Windows Subsystem for linux (WSL2).

\subsection{Building the program}
The project uses \texttt{cmake} as a build system. The following commands shall be run to compile
and install it (it assumes an terminal with the project`s main directory as a working directory):
\begin{enumerate}
  \item \texttt{mkdir build} to create the \texttt{build} directory
  \item \texttt{cmake -B build .} to create the build files in the directory
  \item \texttt{cd build} to change the current working directory to the newly created build
    directory
  \item \texttt{make} for compiling
  \item \texttt{sudo make install} this will ask for your user credentials since it installs the
    program. If you do not trust this program you can also leave this step out. Then you will have
    to change the execution command when running it in the terminal or in the python script
    accordingly. 
\end{enumerate}
Now the program can be run by typing \texttt{microscopicIDM} into the command line, if you did not
install it you need to run \texttt{./microscopicIDM}. You can also add the executable program to the
path of the shell to run it, there are few guides on it in the internet.

\subsection{Running the program and giving instructions}
This program can be executed in the terminal and accepts unix-style command line arguments denoted
with a dash (\texttt{-}).

The simplest execution is simply running \texttt{microscopicIDM}, as it will only run with default
values. \texttt{microscopicIDM -h} will show a help page with command line commands you can add to
specify parameters, e.g. \texttt{microscopicIDM --road-length 5000} to change the road length to
$\SI{5000}{m}$. The most important is the \texttt{--output-csv <filename>} command, as it specifies
the file name and location for the CSV file containing the data.

\section{Differentiation from the previous work}
\label{sec:diff_previous}
During my presentation a question has been raised how my project differs from the previous work. The
biggest differences between MOBIL and this project is
\begin{itemize}
  \item The amount of lanes has been increased to 3
  \item The road is circular
\end{itemize}
While the former is a trivial difference, the second is not. 

Changing the road from an infinite road stretch to a circular one  not only significantly changes the
way data structures and algorithms have to be designed and selected, it also creates a state of
equilibrium. The researchers for the MOBIL publication used and infinite road and had measurements
at fixed locations, after around 5 to 10 kilometers. There was no test if there is a state of
equilibrium so early, but by having a simulated time intervall of over an hour, there results have
been more stable. 

At the same time, it can safe a lot of runtime as less cars are needed and a visual test showed that
for road lengths over 1 km the marginal behaviour stemming from the border can not be observed. A
script creating a clip animation is also given in the project.


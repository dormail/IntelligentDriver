\section{Previous work}
\label{sec:previous_work}
In this report I will cover the previous work only briefly since it has been covered in my midterm report already.
Firstly I will explain the Intelligent Driver Model by Treiber et al. In his work a car following model has been derived
which mimics a driver who is well aware of his surrounding. Afterwards I will give an overview over the lane changing
decision algorithm devoleoped by Kersting et al.

\subsection{The Intelligent Driver Model (IDM)}
\label{sec:idm}
The IDM is a car following model, meaning that a the accelearation of a car is a function of the cars velocity,
distance to the car in front and the velocity of the car in front:
\begin{equation}
  \dot v_\alpha = a \left[1 - \left(\frac{v_\alpha}{v_0^{(\alpha)}}\right)^\delta -
    \left(\frac{s^*(v_\alpha, \Delta v)}{s_\alpha}\right)^2
  \right]
  \label{eqn:idm:sprime}
\end{equation}
It consists of a acceleration term $a [1 - (v_\alpha / v_0^{(\alpha)})^\delta]$ and a term modelling the breaking
behaviour $-a (s^* / s_\alpha)^2$ which gets larger the closer a car drives to the car ahead. The breaking behaviour
can be understood as the ratio between the ``desired minimum gap``
\begin{multline}
  s^*(v, \Delta v) = s_0^{(\alpha)} + s_1^{(\alpha)} \sqrt{\frac{v_\alpha}{v_0^{(\alpha)}}} + T^{(\alpha)}v_\alpha
  \\
  + \frac{v\Delta v}{2 \sqrt{a^{(\alpha)} b^{(\alpha)}}}
\end{multline}
and the actual gap.

\subsection{Minimizing overall braking induced by lane changes (MOBIL)}
\label{sec:mobil}
As explained in the midterm report, there are already few  models for lane change strategies. 
For this project the model 
\textit{Minimizing overall braking induced by lane changes} has been chosen \cite{MOBIL}. 
The model has been developed by
a similar group as the IDM and they proved that they worked well together. 

One key feature of MOBIL is that it contained two modes of driving, symmetric and asymmetric laws with the left
lane as the default lane. This is perfect for our use case as these are the `US` law and Vienna Convention laws
respectively. In the following I will give an overview about the equations for each case.

\subsubsection{Symmetric driving law}
The more simple symmetric driving law, is modeled by the \textit{incentive criterion}
\begin{equation}
  \underbrace{\tilde{a}_c - a_c}_\text{driver}
  + p (
  \underbrace{\tilde{a}_n - a_n}_\text{new follower}
  +
  \underbrace{\tilde{a}_0 - a_0}_\text{old follower}
  )
  > \Delta a_\text{th}.
  \label{eqn:US_incentive}
\end{equation}
While the lower index specifies which car is meant by an expression, the tilde marks a variable as if it a proposed
lane change is accepted, e.g. $\tilde{a}_n$ is the acceleration of the new follower after the car has changed the
lane.

This condition also introduces 2 model parameters, the \textit{politeness factor} $p$ and the switching threshold
$\Delta a_\text{th}$. The former regulates how strong a driver should consider the advantage of a lane change for
the next neighbors. On a qualitative note, $p=0$ indicates a driver ignoring his surrounding, while $p>0$ can be
interpreted as altruistic driving behaviour. Secondly, the lane switching threshold forbids lane changes with
minimal gains.

\subsubsection{Asymmetric driving law}
The asymmetric driving law is more complicated and more complex decision trees. There are two basic rules that
apply here:
\begin{enumerate}
  \item Passing rule: A car must not pass on the right-hand lane. This rule only applies when traffic is not
    congested. Congested traffic is defined as cars moving below a \textit{ciritical velocity} 
    $v_\text{crit} = 60 \,
    \text{km/h}$.
  \item Lane usage rule: A car should only move to the left for overtaking and treat the right lane as the default
    lane.
\end{enumerate}

The passing rule is implemented by using a new variable on the right lane of a lane change:
\begin{equation}
  a_c^\text{eur} = 
  \begin{cases}
    \text{min}(a_c, \tilde{a}_c) & \text{if } v_c > \tilde v_\text{lead} > v_{crit}, \\
    a_c                         & \text{otherwise}
  \end{cases}
\end{equation}
with $\tilde a_c$ for the acceleration on the left lane and $\tilde v_\text{lead}$ corresponding to the velocity of
the car in front on the left lane.
Using $a_c^\text{eur}$ for the dynamics on right-hand lane implements the logic for the critical behaviour around
$v_\text{crit}$.

To implement the lane usage rule, an asymmetry gets added to the incentive criterion. When a change to a right hand
lane is proposed, the inequility reads
\begin{equation}
  L \rightarrow R: \quad
  \tilde a_c^\text{eur} - a_c + p (\tilde a_o - a_o) > \Delta a_\text{th} - \Delta a_\text{bias},
\end{equation}
for changing to the left this inequality becomes
\begin{equation}
  R \rightarrow L: \quad
  \tilde a_c - a_c^\text{eur} + p (\tilde a_n - a_n) > \Delta a_\text{th} + \Delta a_\text{bias}.
\end{equation}
A key difference to \autoref{eqn:US_incentive} is that not all neighbors get considered anymore, as the left lane
(so the new follower for left-bound changes and old follower for changes to the right) gets a priority. Treating
the right lane as the default lane is enforced by the bias $\Delta a_\text{bias}$. With 
$\Delta a_\text{bias} > \Delta a_\text{th}$, a free moving car will change to the right.


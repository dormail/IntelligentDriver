\section{Previous work}
\label{sec:previous_work}
In this report I will cover the previous work only briefly since it has been covered in my midterm report already.
Firstly I will explain the Intelligent Driver Model by Treiber et al. In his work a car following model has been derived
which mimics a driver who is well aware of his surrounding. Afterwards I will give an overview over the lane changing
decision algorithm devoleoped by Kersting et al.

\subsection{The Intelligent Driver Model (IDM)}
\label{sec:idm}
The IDM is a car following model, meaning that a the accelearation of a car is a function of the cars velocity,
distance to the car in front and the velocity of the car in front:
\begin{equation}
  \dot v_\alpha = a \left[1 - \left(\frac{v_\alpha}{v_0^{(\alpha)}}\right)^\delta -
    \left(\frac{s^*(v_\alpha, \Delta v)}{s_\alpha}\right)^2
  \right]
  \label{eqn:idm:sprime}
\end{equation}
It consists of a acceleration term $a [1 - (v_\alpha / v_0^{(\alpha)})^\delta]$ and a term modelling the breaking
behaviour $-a (s^* / s_\alpha)^2$ which gets larger the closer a car drives to the car ahead. The breaking behaviour
can be understood as the ratio between the ``desired minimum gap``
\begin{multline}
  s^*(v, \Delta v) = s_0^{(\alpha)} + s_1^{(\alpha)} \sqrt{\frac{v_\alpha}{v_0^{(\alpha)}}} + T^{(\alpha)}v_\alpha
  \\
  + \frac{v\Delta v}{2 \sqrt{a^{(\alpha)} b^{(\alpha)}}}
\end{multline}
and the actual gap.

\subsection{Minimizing overall braking induced by lane changes (MOBIL)}
\label{sec:mobil}

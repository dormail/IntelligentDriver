\section{Conclusion and Outlook}
\label{sec:conclusion}
In this project, 
\begin{itemize}
  \item The IDM and MOBIL algorithms have been implemented to create microscopic multi lane road
    simulation,
  \item Optimizations to the program have been used to reduce execution time by using pointers and
    multithreaded exections,
  \item A road`s performance has been evaluated regarding speed and throughput for symmetric and
    asymmetric trafic laws,
  \item and tested against a real world test.
\end{itemize}
The results in \autoref{sec:results} showed only a marginal difference between the two mentioned
traffic laws for both speed and throughput. A per-lane analysis showed that when adding more cars
the right has the strongest performance loss in the asymmetric traffic law, for the symmetric case
all lanes have an equal loss.

For future projects more work can be done by
\begin{itemize}
  \item Utilizing more refined data structures as the \texttt{std::vector} can be replaced by a
    double linked list or similar structures as it makes finding neighbors easy,
  \item Improving the data analysis; currently the average speed computation is lacking in
    refinement, as it looks at the peek for the total time integration and does not take variances
    into account,
  \item Better integration techniques can be used as here only forward-euler has been used for
    simplicity,
  \item Using a larger parameter space as most parameters have been assumed equal for
    all cars, which is a strong simplification,
  \item More performance measures can be used as average speed and and throughput are to simple and
    one-dimensional.
\end{itemize}


\section{Previous work and literature}
\label{sec:previos_work}

\subsection{Previous work on car following models}
\label{sec:previous_work:car_following}

In this work, we will use the intelligent driver model (IDM) \cite{PhysRevE.62.1805}. With this model the acceleration
of a car depends on the driving behaviour of the vehicle in front and on the model parameters
\begin{itemize}
  \item Desired velocity $v_0$,
  \item Safe time headaway $T$,
  \item Maximum acceleration $a$,
  \item Desired deceleration $b$,
  \item Acceleration exponent $\delta$,
  \item Jam distance $s_0$,
  \item Jam distance $s_1$,
  \item Vehicle length $l$.
\end{itemize}
When using $s_1 = \SI{0}{m}$ like the original authors, the first jam
distance $s_0$ can be interpreted as the minimum distance a driver wants to the vehicle in front. With those parameters
the acceleration is a continuous function
\begin{equation}
  \dot v_\alpha = a \left[1 - \left(\frac{v_\alpha}{v_0^{(\alpha)}}\right)^\delta -
    \left(\frac{s^*(v_\alpha, \Delta v)}{s_\alpha}\right)^2
  \right].
\end{equation}
It consists of a acceleration term $a [1 - (v_\alpha / v_0^{(\alpha)})^\delta]$ and a term modelling the breaking
behaviour $-a (s^* / s_\alpha)^2$ which gets larger the closer a car drives to the car ahead. The breaking behaviour
can be understood as the ratio between the ``desired minimum gap``
\begin{multline}
  s^*(v, \Delta v) = s_0^{(\alpha)} + s_1^{(\alpha)} \sqrt{\frac{v_\alpha}{v_0^{(\alpha)}}} + T^{(\alpha)}v_\alpha
  \\
  + \frac{v\Delta v}{2 \sqrt{a^{(\alpha)} b^{(\alpha)}}}
\end{multline}
and the actual gap $s_\alpha$ squared. 

This model has been used extensively to model car behaviour under varying conditions \cite{FIDM,CAPRANI2016207} and to optimize
driver assistance systems \cite{thesis_acc}.

\subsection{Generalization to multiple lanes}
\label{sec:previous_work:multiple_lanes}
The model introduced in \autoref{sec:previous_work:car_following} by itself is only a single lane model, since it does
not take into account anything besides the car in front of each other car, i.e. it multiple lanes do not interact. A
solution was proposed bei Kesting et al \cite{MOBIL} with the \textit{Minimizing Overall Breaking Induced by Lane
Changes} (MOBIL) which introduced incentives and sefaty criterions to make decisions about when a car should change a
lane.

The safety criterion ensures that a car following the lane changer would only have to decelerate by a maximum
deceleration $\dot v_{\alpha + 1}^\prime \geq -b$, the parameter $b$ does not have to be the desired deceleration from
the IDM however.

While the safety criterion only forbids lane changes, the incentive criterion makes them possible. The incentive
computes the impact the lane change has to the car`s surrounding and if itself or other people would benefit from it.
For that purpose a \textit{politeness factor} $p$ and a threshold acceleration $\Delta a_\text{th}$ get introduced. 
The condition is a basic inequality
\begin{equation}
  \underbrace{\tilde{a}_c - a_c}_\text{driver}
  + p (
  \underbrace{\tilde{a}_n - a_n}_\text{new follower}
  +
  \underbrace{\tilde{a}_0 - a_0}_\text{old follower}
  )
  > \Delta a_\text{th}
\end{equation}
and the model accepts the proposed lane change when it holds. While the impact of $\Delta a_\text{th}$ is to forbid lane
changes with minor benefits, $p$ has more implications on the driver`s behaviour: As it regulates how much a driver
considers his impact on other people on the road, $p=0$ would make him ignore other people and act only for his own
benefit. $p>0$ implies that a driver accepts getting slowed down after a lane change so that other drivers could
benefit (e.g. changing lane so another driver can pass you). Interestingly, $p<0$ is also possible. That would result in
malicous drivers, drivers who slow themself down with a maneuver for the sake of slowing down other people even
more. 

In the original MOBIL paper the authors also differentiated between the two traffic law types, which will be used in
this project as well.


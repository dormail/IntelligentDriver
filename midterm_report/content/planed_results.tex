\section{Results}
\label{sec:results}
When it comes to a road`s performance, it depends on many variables, like the traffic density and the type of vehicles
on the road. To answer the question proposed in \autoref{sec:goal} we will vary some of those parameters and 
compare the two
passing models regarding their performance. The performance can be measured in terms of macroscopic variable like
average speed, average $v^{(\alpha)}/v_0^{(\alpha)}$ or car throughput. 
Repeating the measurement for many pairs of parameters will allow us to predict when which model is more effective and 
find critical points, boundries between parameter spaces where a specific model performs better. In
\autoref{fig:example_visualization} an example is shown how the two model could be compared regarding one parameter and
one observable, with a critical point marked.
\begin{figure}[H]
	\centering
	\includegraphics[width=0.9\linewidth]{build/example_plot.pdf}
	\caption{Example visualization.}
	\label{fig:example_visualization}
\end{figure}


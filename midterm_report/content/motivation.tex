\section{Motivation and Goal of this project}
\label{sec:goal}
Looking at traffic laws in different parts of the world, regarding passing there are two main types. One candidate
is the driving law as it is implemented in many European countries, which are summed up in the United
Nations Vienna convention about road traffic \cite{vienna-treaty}. Here a driver is obligated to stay as far from the
center of the road when he is not overtaking another driver and shall move towards the center when he is overtaking.
Opposed to this there are countries, like some US states, where a driver is free to choose the lane for any maneuver.
What this means, is that a car can be legally passed on both sites. In Germany, where traffic law obeys the Vienna
convention, both the person passing on the right and the person being passed can actually be prosecuted (see §2 STVO in 
German law \cite{STVO2}).

With those two models being widely implemented, one can raise the question which one performs better. With road traffic, the
term ``performance`` is ambiguous:
\begin{itemize}
  \item Road safety (e.g. accidents, fatalities),
  \item Individual travel time,
  \item Overall car throughput,
  \item Overall emissions
\end{itemize}
are all valid discplines a road`s performance can be measured in.

In this work, I will focus on the individual travel time and the overall car throughput. So the question I raise is:
\newline
How does the obligation to drive on a specific side of the road impact car thoughput and speed?

